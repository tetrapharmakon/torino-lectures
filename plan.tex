% Created 2018-11-12 lun 15:37
% Intended LaTeX compiler: pdflatex
\documentclass[11pt]{article}
\usepackage[utf8]{inputenc}
\usepackage[T1]{fontenc}
\usepackage{graphicx}
\usepackage{grffile}
\usepackage{longtable}
\usepackage{wrapfig}
\usepackage{rotating}
\usepackage[normalem]{ulem}
\usepackage{amsmath}
\usepackage{textcomp}
\usepackage{amssymb}
\usepackage{capt-of}
\usepackage{hyperref}
\usepackage{graphicx}
\usepackage{eucal, hyphenat}
\usepackage{tikz-cd}
\hyphenation{mo-no-i-da-le}
\def\C{\mathcal{C}}
\usepackage[all,2cell]{xy}\UseAllTwocells
\def\Cat{\mathsf{Cat}}
\def\Set{\mathsf{Set}}
\def\xto#1{\xrightarrow{#1}}
\def\xot#1{\xleftarrow{#1}}
\def\To{\Rightarrow}
\usepackage[all,2cell]{xy}
\newcommand{\deduction}[4]{\begin{array}{c} #1 \to #2 \\ \hline #3 \to #4 \end{array}}
\newcommand{\Nearrow}{\rotatebox[origin=c]{45}{$\Rightarrow$}}  % ↗
\newcommand{\Nwarrow}{\rotatebox[origin=c]{135}{$\Rightarrow$}} % ↖
\newcommand{\Searrow}{\rotatebox[origin=c]{-45}{$\Rightarrow$}} % ↘
\newcommand{\Swarrow}{\rotatebox[origin=c]{225}{$\Rightarrow$}} % ↙
\newcommand{\Sarrow}{\rotatebox[origin=c] {-90}{$\Rightarrow$}}
\newcommand{\Narrow}{\rotatebox[origin=c] {90}{$\Rightarrow$}}
\usepackage{turnstile}
\newcommand{\adjunct}[2]{\nsststile{#2}{#1}}
\def\opp{\mathrm{op}}
\def\co{\mathrm{co}}
\def\coop{\mathrm{coop}}
\def\rift{\mathrm{rift}}
\def\leeft{\mathrm{lift}} % `lift is already something!
\def\lan{\mathrm{lan}}
\def\ran{\mathrm{ran}}
\def\Rift{\mathrm{Rift}}
\def\Lift{\mathrm{Lift}}
\def\Ran{\mathrm{Ran}}
\def\Lan{\mathrm{Lan}}
\def\RIFT{\textsc{rift}}
\def\LIFT{\textsc{lift}}
\def\RAN{\textsc{ran}}
\def\LAN{\textsc{lan}}
\usepackage{amsthm}
\theoremstyle{reference}
\newtheorem{theorem}{Theorem}[section]
\newtheorem{definition}[theorem]{Definizione}
\newtheorem{axiom}[theorem]{Assioma}
\newtheorem{lemma}[theorem]{Lemma}
\newtheorem{proposition}[theorem]{Proposizione}
\newtheorem{remark}[theorem]{Osservazione}
\hypersetup{colorlinks=true, linkcolor=black}
\renewcommand{\bibname}{Alcuni riferimenti}
\author{Fosco Loregian}
\date{\today}
\title{Strutture di Yoneda e accessibilità}
\hypersetup{
 pdfauthor={Fosco Loregian},
 pdftitle={Strutture di Yoneda e accessibilità},
 pdfkeywords={},
 pdfsubject={},
 pdfcreator={Emacs 25.2.2 (Org mode 9.1.3)}, 
 pdflang={English}}
\begin{document}

\maketitle
\tableofcontents

\input{the-bib.bbl}

\section{Lecture I: Introduzione}
\label{sec:org2468683}
\subsection{2-categorie: def di partenza}
\label{sec:orgdb57cf0}
Una \emph{2-categoria} è ``come una categoria, ma gli hom-set
sono categorie'': si tratta di un certo tipo di struttura
2-dimensionale che soddisfa le seguenti ipotesi

\begin{enumerate}
\item E' data una classe di oggetti \(\mathcal K_0\),
\item E' dato per ogni coppia di oggetti \(X,Y\in \cal K_0\) un
insieme di morfismi (o \emph{1-celle}) \({\cal K}(X,Y)\)
\item E' data una regola di composizione
\end{enumerate}
$$ c_{XYZ} : {\cal K}(X,Y)\times {\cal K}(Y,Z) \to {\cal
K}(X,Z) $$ che sia \emph{bifuntoriale}, ossia tale esista una
regola di composizione orizzontale, una verticale, e tali
che per esse valga la ``regola di interscambio di
Godement'': $$ \xymatrix{A \ruppertwocell{\beta}
\ar[r]\rlowertwocell{\delta}& B \ruppertwocell{\alpha}
\ar[r]\rlowertwocell{\gamma}& C} $$ $$ (\alpha *\beta)\circ
(\gamma * \delta) = (\gamma \circ\alpha) * (\delta \circ
\beta)$$ Una 2-categoria è una categoria arricchita su
\(\Cat\), guardata come base monoidale (chiusa): ogni
\(\Cat({\cal C},{\cal D})\) è a sua volta una categoria, i cui
oggetti sono i funtori \(F,G : \C\to {\cal D}\) e i cui
morfismi sono le trasformazioni naturali. Ciascuna
composizione è ``bilineare'', ed esiste una nozione di
funtore arricchito (\emph{2-funtore stretto}) e di trasformazione
naturale arricchita (\emph{trasf. naturale stretta}).

La teoria delle 2-categorie coincide allora con la teoria
delle \(\Cat\) -categorie? In parte sì: un gran numero di
risultati è conseguenza della teoria generale sviluppata nel
primo capitolo. Del resto, una parte ancora maggiore di
risultati non si può ingabbiare nel linguaggio delle
categorie arricchite: e questo perché segretamente la
collezione delle 2-categorie (di cui, fatti salvi alcuni
problemi di teoria degli insiemi, \(\Cat\) è un oggetto) è una
\textbf{3-categoria}; le sue proprietà sono quindi più naturalmente
descritte da strutture e leggi di coerenza più blande di
quelle che sostengono la teoria delle \(\Cat\) -categorie.

Un esempio su tutti: nei termini di una \(\Cat\) -categoria è
complicato (o impossibile?) descrivere come faccia un
diagramma a commutare \emph{a meno di una trasformazione
naturale} -invertibile o meno.

Alcune osservazioni:

\begin{itemize}
\item Ogni 2-categoria dà luogo ad altre 2-categorie \(\mathcal
  K^\co, \mathcal K^\opp\) dove rispettivamente sono state
invertite le frecce in dimensione 2 ed 1. Più formalmente,
\(\mathcal K^\co\) è una 2-categoria ottenuta da \(\mathcal
  K\) che ha gli stessi oggetti e dove \(\mathcal K^\co(X,Y) =
  \mathcal K(X,Y)^\opp\), e \(\mathcal K^\opp\) è una
2-categoria con gli stessi oggetti di \(\mathcal K\), dove
\(\mathcal K^\opp(X,Y) = \mathcal K(Y,X)\). Chiaramente,
\((\mathcal K^\opp)^\co = (\mathcal K^\co)^\opp = \mathcal
  K^\text{coop}\).
\item Gli oggetti di \(\mathcal K\) si dicono \emph{0-celle}; gli
oggetti delle categorie \({\mathcal K}(X,Y)\) si dicono
1-celle (di dominio \(X\) e codominio \(Y\)); i morfismi delle
categorie \({\mathcal K}(X,Y)\) si dicono 2-celle di
\(\mathcal K\).
\end{itemize}

\subsection{Estensioni e lift di Kan}
\label{sec:org15cd52d}

\begin{definition}
Let $B \xto{f} A \xot{g}C$ a cospan of
1-cells in ${\mathcal K}$. A /left lifting/ of $f$ along $g$
consists of a pair $\langle\leeft_gf,\eta\rangle$ (often
denoted simply as $\leeft_gf$) initial among the commutative
triangles like the one below: 
\[
\vcenter{\xymatrix@C=1.4cm{& C\ar[d]^g \\ B\ar[r]_f
\ar@{.>}[ur]^{\leeft_gf} & \ar@{}[ul]|(.3){\Nearrow\eta} A}}
\qquad \deduction{\leeft_gf}{h}{f}{gh} 
\] In other words,
composition with $\eta \colon f \To g \circ \leeft_gf$
determines a bijection $\bar\gamma \mapsto (g *
\bar\gamma)\circ \eta$ between 2-cells $\leeft_gf
\xto{\bar\gamma} h$ and 2-cells $f \to gh$.
\end{definition}

Si può anche definire un \emph{lifting destro} in maniera
simile, rovesciando la direzione delle sole 2-celle nel
diagramma precedente, e le \emph{estensioni sinistre} e
\emph{destre} rispettivamente rovesciando solo la direzione
delle 1-celle, o la direzione di entrambe nel diagramma
precedente. E' quindi chiaro che le estensioni sinistre sono
lifting sinistri in \({\mathcal K}^\opp\), i lifting destri in
\({\mathcal K}\) sono lifting sinistri in \({\mathcal K}^\co\),
e le estensioni destre sono lifting sinistri in \({\mathcal
K}^\coop\).

\begin{center}
\begin{array}{|c|c|}\hline \xymatrix{A \ar@{}[dr]|(.3){\Swarrow\eta}\ar[d]_g
\ar[r]^f& B \\ C \ar@{.>}[ur]_{\Lan_gf} & {\tiny \deduction{\Lan_gf}{h}{f}{hg}}}
& \xymatrix{{\tiny \deduction{\Lift_gf}{h}{f}{gh}} & C\ar[d]^g \\ B\ar[r]_f
\ar@{.>}[ur]^{\Lift_gf} & \ar@{}[ul]|(.3){\Nearrow\eta} A} \\ \hline
%%%
\xymatrix{A \ar@{}[dr]|(.3){\Nearrow\varepsilon}\ar[d]_g \ar[r]^f& B \\ C
\ar@{.>}[ur]_{\Ran_gf} & {\tiny \deduction{hg}{f}{h}{\Ran_gf}}} &
\xymatrix{{\tiny \deduction{h}{\Rift_gf}{gH}{f}} & C\ar[d]^g \\ B\ar[r]_f
\ar@{.>}[ur]^{\Rift_gf} & \ar@{}[ul]|(.3){\Swarrow\varepsilon} A} \\ \hline
\end{array}
\end{center}

\begin{definition}[Estensione/lift preservato/assoluto]

\end{definition}

\subsection{Caratterizzazione degli aggiunti mediante estensioni e lift}
\label{sec:org693d804}

Qui ci proponiamo di dimostrare una caratterizzazione degli
aggiunti in termini di lift ed estensioni. Si tratta di un
esercizio elementare nelle proprietà universali in una
2-categoria, che svolgiamo nei dettagli per fare entrare il
lettore nella semantica delle 2-categorie.

Le seguenti condizioni sono equivalenti per una coppia di
1-celle \(f : A \leftrightarrows B : g\):
\begin{itemize}
\item  $f \dashv g$ con unità $\eta$ e counità $\epsilon$;
\item  La coppia $\langle g,\eta\rangle$ esibisce la Lan assoluta di $1$ lungo $f$
\item  La coppia $\langle g,\eta\rangle$ esibisce la Lan di $1$ lungo $f$, ed $f$ la preserva.
\end{itemize}

\begin{proof}
E' evidente che 2 implica 3; mostriamo che 1 implica 2. Dato il diagramma
$$
\xymatrix{
A \ar@{=}[r]\ar@{}[dr]|(.3){\Swarrow\eta}\ar[d]_f & A  \\
B \ar[ur]_g & 
}
$$
dobbiamo mostrare che è una Lan assoluta. Del resto,  se $f \dashv g$, dato $h : B\to A$ con una trasformazione $\alpha : 1\To hf$, le identità triangolari implicano che la composizione $\bar\alpha : g \overset{\alpha * g} \To hfg \overset{h * \epsilon}\To h$ sia tale che $(\bar \alpha * f)\circ \eta = \alpha$. Tale scelta è unica, perché se $\bar\alpha$ e $\hat\alpha$ hanno la stessa proprietà, basta incollare la counità per vedere che $\bar \alpha * g = \hat\alpha * g$:
$$
\vcenter{\xymatrix{
& A \rrlowertwocell<\omit>{<3>\eta} \ar[dr]_f\ar@{=}[rr] & & A \\
B \rruppertwocell<\omit>{<-3>\epsilon} \ar[ur]^g\ar@{=}[rr] && B \ar[ur]^g\urlowertwocell{\bar\alpha} & 
}}
\quad = \quad
\vcenter{\xymatrix{
& A \rrlowertwocell<\omit>{<3>\eta} \ar[dr]_f\ar@{=}[rr] & & A \\
B \rruppertwocell<\omit>{<-3>\epsilon} \ar[ur]^g\ar@{=}[rr] && B \ar[ur]^g\urlowertwocell{\hat\alpha} & 
}}
$$
Un argomento simile mostra che l'estensione è assoluta: dato un diagramma come
\[
\xymatrix{
A \ar@{=}[r]\ar[d]_f & A \ar[r]^u  & X \\
B  \ar@/_1pc/[urr]_h \ar[ur]_g& &
}
\]
riempito da una 2-cella $\alpha : u \To hf$, va mostrato che esiste un'unica $\bar\alpha : ug\To h$ tale che $\alpha = (\bar\alpha * f)\circ(u * \eta)$. Tale freccia è presto vista essere $(h * \epsilon)\circ(\alpha *g)$.

Ora mostriamo che 3 implica 1. Se $\langle fg, f *\eta\rangle$ esibisce $\lan_ff$, allora è automatico che esista un'unica $\epsilon : fg\To 1$ tale che $(\epsilon * f)\circ (f * \eta) = 1_f$; per quanto riguarda l'altra identità triangolare\dots
\end{proof}
Valgono condizioni duali che si possono riassumere nel seguente specchietto:
\begin{center}
\includegraphics{adjs}
\end{center}

\subsection{Il sacro pasting lemma}
\label{sec:org897dd43}

Uno dei trucchi tecnici più utili è il seguente: si chiama
\emph{pasting lemma}. La dimostrazione si fa verificando la
proprietà universale giusta.
\begin{proposition}
Dato un diagramma come
$$
\begin{tikzcd}
|[alias=a]|A \ar[r,"h"]\ar[d,"f"']&|[alias=d]| D &|[alias=a']| A \ar[d]\ar[r,"h"]&|[alias=d']| D \\
B \ar[d,"g"']\ar[ur]&&|[alias=b']| B \ar[d]\ar[ur]\\
|[alias=c]|C \ar[bend right,uur] && C\ar[bend right,uur] 
\end{tikzcd}
$$
se il triangolo esterno e quello superiore sono estensioni di Kan, tale è anche il rimanente triangolo.
\end{proposition}
\begin{proof}
  Esercizio.
\end{proof}
\subsection{Strutture di Yoneda e FCT: il lemma di Yoneda come pilastro del pensiero occidentale}
\label{sec:org55a4f6d}

\subsubsection{Yoneda nel senso classico; Yoneda "per i baby geometri"}
\label{sec:org8062c05}

Il lemma di Yoneda come viene insegnato nelle scuole
inferiori, asserisce che dato un prefascio \(F : \C^\opp \to
\Set\) e un oggetto \(c\in\ C\) esiste un isomorfismo naturale
in \(c\) $$ Nat(y(c), F)\cong Fc $$ dove \(y : \C \to
[\C^\opp,\Set]\) è l'embedding di Yoneda che manda \(c\) nel
prefascio \(\C(-,c)\).

Ora, lungi dall’essere un mero teorema di matemaitca, questo
asserto costituisce uno dei punti più elevati raggiunti dal
pensiero occidentale nella sua totalità. Diverse generazioni
di studio sono state completamente insufficienti a
disvelarne le incredibili conseguenze.

Quello che facciamo noi ora è

\begin{itemize}
\item Scrivere il lemma di Yoneda in forma fibrazionale;
\item Capire in che cosa consiste quando i prefasci vengono
interpretati come fibrazioni
\end{itemize}

Per farlo ci avvaliamo di questo risultato:
\begin{proposition}
Esiste un’equcat tra $[\C^\opp,\Set]$ (la categoria dei
prefasci su $\C$) e la categoria delle \emph{fibrazioni discrete}
su $\C$ (una fibrazione discreta è un funtore $p : \mathcal
E \to \C$ tale che ogni fibra $p^\leftarrow(c)$ sia una
categoria discreta). 
\end{proposition}
\begin{proof}
E' sufficiente dimostrare che esiste una coppia di funtori
in direzioni opposte 
$$ [\C^\opp,\Set] \leftrightarrows
\text{DFib}(\C) $$ 
le cui composizioni nei due sensi siano
isomorfe alle rispettive identità (perché?). Per farlo,
definiamo ${\mathfrak E} : [\C^\opp,\Set] \to
\text{DFib}(\C)$ mandando $P$ nella sua categoria degli
elementi; in direzione opposta, definiamo ${\mathfrak F} :
\text{DFib}(\C) \to [\C^\opp,\Set]$ mandando $p : \mathcal
E\to \C$ nel prefascio determinato da $\lambda
c.p^\leftarrow(c)$ (dal momento che $p$ è una fibrazione
discreta, questa corrispondenza è davvero un funtore). E'
evidente che $\mathfrak{EF}\cong 1$, così come
$\mathfrak{FE}\cong 1$.
\end{proof}
In tale contesto il lemma di Yoneda diventa il seguente enunciato:
\begin{lemma}[Yoneda fibrazionale]
C'è una biiezione
$$\left\{
{\small 
\vcenter{
  \xymatrix@!=3mm{
  \C/c \ar@{.>}[rr]\ar[dr]_U && \mathfrak E(P)\ar[dl]^\pi \\
  & \C & 
  }
}}
\right\} \cong Pc$$
tra le frecce tratteggiate e gli elementi di $Pc$.
\end{lemma}
\begin{proof}
Esercizio.
\end{proof}

\subsubsection{Di cosa parliamo quando parliamo di teoria delle categorie?}
\label{sec:org0c00629}

Che cos’è la teoria delle categorie? Nelle parole di John Gray,
\begin{quote}
The purpose of category theory is to try to describe certain general
aspects of the structure of mathematics. Since category theory
is also part of mathematics, this categorical type of description
should apply to it as well as to other parts of mathematics.

[O]ne should attempt to identify those properties that enable one
to do the "structural parts of category theory".
\end{quote}
Che cosa significa questo? Sostanzialmente che la teoria
delle categorie "astratta" è quell’insieme di asserti che
riguardano il comportamento di strutture che si comportano
come la 2-categoria paradigmatica \(\Cat\) (allo stesso modo,
la teoria delle categorie "concreta" consta di quegli
asserti che riguardano categorie che si comportano come
quelle di oggetti matematici quotidiani: la categoria degli
insiemi, quella dei gruppi abeliani, dei monoidi, degli
insiemi o spazi vettoriali con una azione di gruppo. . . ).
Analogamente a quel che succede quando si usa la teoria
delle categorie per chiarificare la matematica classica (in
modo che le proprietà degli oggetti matematici diventino
proprietà universali, e che queste proprietà universali
siano godute dagli oggetti di una categoria, definendo, ad
esempio, la semantica funtoriale delle teorie algebriche),
la teoria delle 2-categorie fa lo stesso lavoro con la
teoria delle categorie. Alle entità fondamentali della
teoria delle categorie (gli aggiunti e il loro calcolo, le
monadi, le estensioni di Kan, il calcolo dei co/limiti\dots{})
viene data licenza di esistere non più nella 2-categoria
\(\Cat\), ma in una generica 2-categoria \(\mathcal K\). La
nozione di struttura di Yoneda nasce per dare conto di
queste affermazioni e concretizzarle in una teoria esplicita
e computabile: prendiamo come assiomi fondamentali di questa
religione il fatto che

\begin{itemize}
\item la teoria delle categorie coincide con l’insieme dei
corollari del lemma di Yoneda;
\item E’ possibile enunciare un insieme finito di assiomi capaci
di catturare le varie facce del lemma di Yoneda;
\item L’intero comparto di tecniche della CT formale nasce per
rispondere a questa domanda: qual è il minimo amount di
struttura addizionale da mettere su una 2-categoria
\(\mathcal K\) per fare in modo che esistano, in \(\mathcal
  K\), delle 1-celle che giocano lo stesso ruolo delle
fibrazioni discrete, dando a \(\mathcal K\) una versione
fibrazionale del lemma di Yoneda?
\end{itemize}

\subsection{Assiomi di struttura di Yoneda}
\label{sec:org1708580}

Gli assiomi sono 4. Seguiamo questo pattern:

\begin{itemize}
\item Come zeresimo passo, listiamo i dati che K deve possedere;
questi dati formano un telaio di Yoneda.
\item Prima enunciamo l’assioma;
\item Poi mostriamo perché è vero in \(\Cat\) (la risposta sarà
che l’assioma è una conseguenza del lemma di Yoneda, in un
modo o nell’altro);
\item Poi enucleiamo alcuni corollari di quell’assioma.
\end{itemize}

\begin{definition}
Affinché K abbia un telaio di Yoneda essa deve essere
equipaggiata di questi dati:
\begin{itemize}
\item Un ideale di morfismi "ammissibili"; le frecce identiche nell’ideale specificano gli \emph{oggetti} ammissibili.
\item Per ogni oggetto ammissibile $A$ una "freccia di Yoneda" $y_A : A \to P A$ verso un oggetto che chiamiamo "oggetto dei prefasci" di $A$;
\item per ogni morfismo ammissibile $f : A\to B$ con dominio ammissibile un triangolo
$$
\xymatrix{
  &A \ar[dr]^f\ar[dl]_{y_A}&\\
PA \urlowertwocell<\omit>{<3>\quad\chi^f}&&\ar[ll]^{B(f,1)} B
}
$$
\end{itemize}
\end{definition}
\begin{axiom}
La coppia $\langle B(f,1), \chi^f\rangle$ esibisce $\lan_fy_A$.
\end{axiom}
Perché è vero in \(\Cat\)? E' il lemma di Yoneda, nella forma
che asserisce l'esistenza di un funtore \(N_f = B(f,1) :
\lambda b.(\lambda a. B(fa,b))\), detto \emph{$f$ nervo}. Ad
esempio, quando \(f : \Delta \to \Cat\) è il funtore che
realizza ogni ordinale finito come una categoria,
\(\Cat(f,A)\) è il \emph{nervo} della categoria \(A\in\Cat\),
che manda \(n\) in \(\Cat([n],A)\). 
\begin{proof}
Bisogna mostrare l'isomorfismo $$[B,PA](N_f,G) \cong
[A,PA](y_A,G\circ f).$$
Per farlo, è sufficiente considerare l'isomorfismo integrale
\begin{align*}
[B,PA](N_f,G) &\cong \int_b PA(B(f,b),Gb)\\
&\cong \int_{ab} \Set(B(fa,b), G(b)(a))\\
&\cong G(fa)(a)\\
[A,PA](y_A,G\circ f) & \cong \int_a PA(y_A(a), G(fa))\\
&\cong G(fa)(a).
\end{align*}
E' ovviamente possibile una dimostrazione diretta, con la proprietà universale: la lasciamo come esercizio.
\end{proof}
\begin{axiom}
La coppia $\langle f, \chi^f\rangle$ esibisce $\leeft_{B(f,1)}y_A$.
\end{axiom}
Perché è vero in \(\Cat\)? E' il lemma di Yoneda, nella forma che asserisce che vale l'isomorfismo
\begin{align*} 
[A,PA]\big( y_A, N_f\circ g \big) &\cong \int_{a'}[A^\opp,\Set]\big(y_A{a'}, N_f\circ g(a')\big)\\ 
& \cong \int_{a'}[A^\opp,\Set]\big( y_A{a'}, B(f - ,ga')\big)\\ 
&\cong \int_{a'}B(fa',ga')\\ &\cong [A,B](f,g)
\end{align*}
E' ovviamente possibile una dimostrazione diretta, con la proprietà universale: la lasciamo come esercizio.
\begin{axiom}
Given a pair of composable 1-cells $A \xto{f} B\xto{g} C$, the
pasting of 2-cells
$$ \begin{tikzcd}[column sep=large, row sep=large] A\ar[d, "f"']\ar[rr, "y_A"{name=yonA}] && P A\\ B \ar[r, "y_B"{name=yonB}]\ar[d, "g"'] & P B\ar[ur, "P f"']\\ C\ar[ur, "{C(g,1)}"'] \ar[from=yonA, to=yonB, shorten >=2mm, shorten <=4mm, Rightarrow, "\chi^{y_B f}"] \ar[from=yonB, shorten >=4mm, shorten <=4mm, Rightarrow, "\chi^g"] \end{tikzcd} $$
exhibits $\lan_{gf}y_A = C(gf,1)$.
\end{axiom}
Perché è vero in \(\Cat\)? Pasting lemma delle estensioni.
\begin{axiom}
La coppia $\langle 1_{PA}, 1_{y_A} \rangle$ esibisce $\lan_{y_A}y_A$.
\end{axiom}
Perché è vero in \(\Cat\)? E' il lemma di Yoneda, nella forma che asserisce che l'embedding di Yoneda è un funtore \emph{denso}.
\begin{proof}
Si potrebbe fare con gli integrali usando ancora la formula
puntuale per le Lan, ma una dimostrazione diretta è
illuminante. Srotolando la proprietà universale, va
dimostrato che una trasformazione naturale $\alpha : 1_{PA}
\To H$ è univocamente determinata dalla sua restrizione alle
componenti rappresentabili (nell'immagine essenziale di
$y_A$, che sappiamo già essere pienamente fedele). Ora, data
una $\beta : y_A\To Hy_A$ dobbiamo dimostrare che essa è
$\alpha *y_A$ per un'unica $\alpha : 1\To H$; per farlo
possiamo ricordare che ogni $P : A^\opp\to \Set$ è colimite
di rappresentabili, e precisamente $P \cong
\varinjlim\!{}^Py_A$; sicché la componente di $\beta$ si può
estendere a
$$
P \cong \varinjlim\!{}^P y_A \xto{\varinjlim{}^P\beta} \varinjlim\!{}^P H y_A \to H(\varinjlim\!{}^P  y_A ) \cong HP
$$
Che queste siano le componenti di una trasformazione naturale $1 \To H$ è presto verificato.
\end{proof}


Ora possiamo definire diverse nozioni che non avevano un
analogo controllabile prima di scoprire che \(\mathcal K\)
supportava una teoria delle categorie.

\begin{definition}[Estensioni puntuali e assolute]
Blah 
\end{definition}
\begin{definition}[Caratterizzazione puntuale degli aggiunti]
Blah 
\end{definition}

\subsection{Esempi}
\label{sec:orgbb0cebf}

Raccogliamo degli esempi di strutture di Yoneda

\subsubsection{Categorie arricchite}
\label{sec:org60c4d9b}

Sulla categoria \(\mathcal V\text{-Cat}\) delle categorie
  arricchite su una base monoidale \(\mathcal V\) c'è una
  struttura di Yoneda dove \(y_A\) è la versione arricchita
  dell'embedding classico. (Esiste qualcosa di analogo anche
  per categorie arricchite su una bicategoria?)

\subsubsection{Categorie interne}
\label{sec:org92413a0}

Nella 2-categoria delle categorie interne a \(\mathcal K\)
  (una categoria con limiti finiti, o con almeno pullback)
  c'è una struttura di Yoneda ma fatta con le fibrazioni; ne
  sketchiamo l'esistenza. Let \(\mathcal E\) be a finitely
  complete category, and \(\mathcal K = \Cat(\mathcal E)\) the
  2-category of categories internal to \(\mathcal E\). Recall
  the definition of an internal profunctor; prove that there
  is an equivalence $$ \mathrm{Prof}_{\mathcal E}(A,B) \cong
  \mathrm{Prof}_{\mathcal E}(1,A^\opp\times B)$$ Prove that
  this correspondence is natural in \(A,B\) (which covariance
  type is it?).

We define

\begin{itemize}
\item an \emph{internal full subcategory} of \(\mathcal E\) an
object \(\mathrm{S}\) of \(\mathcal K\) with an internal
profunctor \(s : 1 \rightsquigarrow \mathrm{S}\) inducing a
fully faithful functor \[\mathcal K(X,\mathrm{S}) \to
\mathrm{Prof}_{\mathcal E}(1,B)\] via precomposition. \item
a 1-cell \(f : A\to B\) in \(\mathcal K\) \emph{admissible} when
the profunctor corresponding to \((f/B)\) lies in the
essential image of the functor \(\mathcal K(A^\opp\times
B,\mathrm{S}) \to \mathrm{Prof}_{\mathcal E}(1,A^\opp\times
B)\). call \(f^*\) this (unique) 1-cell \(A^\opp\times B \to
\mathrm{S}\). \end{itemize} Prove that \(\mathcal K\) has a
Yoneda structure when \(B(f,1) := \widehat{f^*} : B \to
[A^\opp,\mathrm{S}]\) is the mate of \(f^*\), and thus \(P A :=
[A^\opp,\mathrm{S}]\).

What happens when \(\mathcal E\) is an elementary topos and
\(\mathrm{S}=\Omega_{\mathcal E}\)? What happens when
\(\mathcal E\) is a Grothendieck topos and \(\mathrm S =
\mathbb{N}\) is the natural number object of \(\mathcal E\)?.

\(\rhd\) Domanda aperta: esiste un teorema che "trasporta" una
struttura di Yoneda lungo l'aggiunzione $$ \mathcal K^{(T)}
\leftrightarrows \mathcal K $$ associata a una 2-monade su
\(\mathcal K\)? (Già trattare il caso idempotente sarebbe
bello)

\subsubsection{Pseudofuntori e derivatori (?)}
\label{sec:org56b2f4c}

Sulla 2-categoria degli pseudofuntori \(\mathcal A \to \Cat\)
(\(\mathcal A\) una bicategoria a caso) c'è una struttura di
Yoneda "puntuale", definita da una opportuna contorsione
fibrazionale.

\(\rhd\) Domanda aperta: questa struttura di Yoneda si riporta
alla 2-categoria (stretta) dei funtori \emph{stretti} \(\mathcal A
\to \Cat\) (ora \(\mathcal A\) è una 2-categoria stretta). Come
si trova una struttura di Yoneda sulla 2-categoria dei
prederivatori che abbia un \emph{significato omotopico}?

\subsection{La nozione di P-cocompletezza}
\label{sec:org70d3bdf}

Negli assiomi di YS è nascosto il fatto che \(PA\) ha la
proprietà universale del \emph{cocompletamento libero} di \(A\):
dove? E' magari possibile dimostrare che un oggetto \(X\) è
cocompleto se e solo se tutte le 1-celle \(A\to X\) si
estendono a una aggiunzione \(PA \leftrightarrows X\)? La
risposta è sì, ma affinché sia vero \(PA\) deve essere
"cocompleto rispetto a sé stesso".

\begin{definition}[Oggetto co/completo]
Un oggetto $X\in \mathcal K$ si dice $P$\hyp{}\emph{cocompleto} quando in ogni diagramma
$$
\xymatrix{
& PG\ar@{.>}[dr]^L & \\
G\ar[rr]_\ell\ar[ur]^{y_A} &\ar@{}[u]|\Uparrow & X
}
$$
la freccia tratteggiata esiste, e la 2-cella rende il triangolo così ottenuto una estensione puntuale.
\end{definition}
\begin{proposition}
Un oggetto $X$ è $P$\hyp{}cocompleto se e solo se esiste una aggiunzione
$$
L : PX \leftrightarrows X : i 
$$
con counità invertibile (quindi se e solo se $X$ "è riflessivo in $PX$").
\end{proposition}

\subsection{La vera natura di \(P\).}
\label{sec:orgda23700}

La costruzione dei prefasci \(P\) di una struttura di Yoneda si caratterizza con queste proprietà:

\begin{enumerate}
\item E' tale che ogni \(Pf\) ha un aggiunto sinistro \(P_!f\);
\item La corrispondenza \(f\mapsto P_!f\) definisce una pseudomonade (come una monade, ma uno pseudofuntore) e \emph{relativa} (come una monade, ma non è un endofuntore);
\item Tale monade è \emph{lax idempotente}, ossia le sue algebre \(a : PA\to A\) sono univocamente caratterizzate dal dare una aggiunzione
\end{enumerate}
\[
a : PA \leftrightarrows A : y_A
\]
In un recente lavoro con I. Di Liberti mostriamo che \emph{tutte} le strutture di Yoneda cocomplete nascono a questo modo.

\section{Lecture II: Accessibility and Presentability in 2-categories}
\label{sec:org8df2217}

\subsection{Cosa vogliamo fare}
\label{sec:org824d2fc}

Le categorie accessibili e presentabili sono particolari
oggetti della 2-categoria \(\Cat\); fino a che punto è
possibile sketchare una definizione per un oggetto
accessibile/presentabile di una 2-categoria \(\mathcal K\)? E'
ancora possibile recuperare i teoremi classici di
rappresentazione, che dicono come gli oggetti accessibili
nascano da riflessioni di oggetti dei prefasci?

E' ancora possibile enunciare e dimostrare la dualità di
Gabriel-Ulmer?

\subsection{L'idea per farlo}
\label{sec:org9fc038d}

Utilizzare il linguaggio delle strutture di Yoneda;
idealmente, un oggetto di \(\mathcal K\) sarà presentabile se
nasce da una localizzazione riflessiva di \(PA\) per qualche
oggetto \(A\), e sarà accessibile se è della forma
\(Ind_\alpha(G)\) per qualche "generatore" \(G\). Ci sono però
vari problemi:

\begin{enumerate}
\item Che ruolo hanno i numeri cardinali in questa definizione?
\end{enumerate}
La risposta sarà: sono importanti, ma possiamo
nasconderli sotto il tappeto.
\begin{enumerate}
\item Qual è il significato di "accessibile" in un contesto
\end{enumerate}
dove non si può guardare dentro gli oggetti, e quindi non
si può dire che "esiste un generatore"?

\subsection{Categorie accessibili e presentabili, classicamente.}
\label{sec:org50d9e4a}

\begin{definition}[Categoria accessibile]
Una categoria $\mathcal K$ si dice $\alpha$\hyp{}accessibile se 
\begin{itemize}
\item Ha i colimiti $\alpha$\hyp{}diretti;
\item Ha un insieme di oggetti $\alpha$\hyp{}compatti che genera $\mathcal K$ per colimiti $\alpha$\hyp{}diretti.
\end{itemize}
\end{definition}
\begin{definition}[Categoria presentabile]
Una categoria $\mathcal K$ si dice $\alpha$\hyp{}presentabile se è $\alpha$\hyp{}accessibile e cocompleta.
\end{definition}
\begin{definition}[Teorema di rappresentazione, I]
Equivalent characterizations include that $C$ is $\alpha$\hyp{}accessible iff:
\begin{itemize}
\item it is the category of models (in Set) of some small sketch.
\item it is of the form $Ind_\alpha(S)$ for $S$ small, i.e. the $\alpha$\hyp{}ind-completion of a small category, for some $\alpha$.
\item it is of the form $\alpha\text{-Flat}(S)$ for $S$ small and some $\alpha$, i.e. the category of $\alpha$\hyp{}flat functors from some small category to $Set$.
\end{itemize}
\end{definition}
\begin{definition}[Teorema di rappresentazione, II]
Equivalentemente, una categoria $\mathcal K$ è presentabile se una, e quindi tutte, di queste condizioni è soddisfatta:
\begin{itemize}
\item $\mathcal K$ è la categoria dei modelli di un limit sketch;
\item $\mathcal K$ è equivalente alla categoria $[C,\Set]$ dei funtori da $C$ a $\Set$ che preservano gli $\alpha$\hyp{}limiti;
\item $\mathcal K$ è una localizzazione riflessiva, accessibilmente immersa, di una categoria di prefasci.
\end{itemize}
\end{definition}
\begin{definition}[Dualità di Gabriel-Ulmer]
La dualità di Gabriel-Ulmer asserisce che la 2-categoria
$\sf Lex$ delle categorie con limiti finiti e la 2-categoria
$\sf Prs$ delle categorie $\aleph_0$\hyp{}presentabili sono
biequivalenti (ricorda la definizione di biequivalenza).
\end{definition}
Si tratta di costruire una coppia di funtori
$$
P : \mathsf{Lex} \leftrightarrows \mathsf{Prs} : \mathbf{c}
$$
nel modo seguente: data una categoria \(C\) con limiti finiti,
la sua categoria di prefasci \([C^\opp,\Set]\) è finitamente
presentabile. Viceversa, data una categoria localmente
presentabile \(\mathcal K\), possiamo estrarre la sua
sottocategoria degli oggetti \(\aleph_{\text{0}}\)-compatti; questa è una
categoria con limiti finiti (è un risultato standard), e
questo determina una biaggiunzione.

Ora, le categorie con limiti finiti sono Cauchy-complete,
sicché esse sono univocamente determinate dalle loro
categorie di prefasci; questo significa che \(P\) è pienamente
fedele. Del resto, estrarre la categoria degli oggetti
\(\aleph_{\text{0}}\)-compatti da \(\mathcal K\) è anche lui un funtore
pienamente fedele, perché \(\mathcal K \cong
Ind_\aleph_0(\mathbf{c}(\mathcal K))\).

\subsection{Definizione: Yoneda context}
\label{sec:org59733a9}

\begin{definition}
	A \emph{Yoneda context} is a pseudonatural transformation $y : S \To P$ such that
	\begin{itemize}
		\item for each component $X\in\mathcal K$ the triangle
		\[
			\vcenter{\xymatrix{
			& X \ar[dr]^{\alpha_X}\ar[dl]_{y_X}\ar@{}[d]|{\chi_{P}\Searrow}& \\
			P(X) && \ar[ll]^{y_X} S (X)
			}}
		\]
		exhibits the left extension of $y_X$ along $\alpha_X$, and which is is component-wise representably fully faithful.
		\item The pseudo\hyp{}functor $P\in KZ(\mathcal K)$ underlies a Yoneda structure.
		% \item  $y\in  KZ(\mathcal K)/P$ is a morphism, say $y : S\To P$, which is component-wise $P$ fully faithful.
	\end{itemize}
\end{definition}


\subsection{Definizione: oggetto accessibile wrt un contesto}
\label{sec:orgf6e5502}

\begin{definition}[$y$ accessible object]\label{yonacc}
	Let $y$ be a context on the 2-category $\mathcal K$; $A\in\mathcal K$ is $y$ \emph{accessible} if there exists a $ P $ small object $G\in \mathcal K$ such that $A\cong S  G$.
\end{definition}


\subsection{Definizione: oggetto presentabile wrt un contesto}
\label{sec:org1861897}

resentability is harder to define properly; in fact, the strategy of exploiting a similar intrinsic characterization for the accessibility condition collides with the fact that we are able to give \emph{two} such characterizations:
\begin{itemize}
	\item \label{llp:uno} A category $A$ is locally presentable if it is an accessible, accessibly embedded full reflective subcategory of a category of presheaves.
	\item \label{llp:due} A category $A$ is locally presentable if it is a full reflective subcategory of a category of presheaves such that the inclusion creates $\lambda$ directed colimits.
\end{itemize}
These two characterizations are equivalent in \(\Cat\), but can't be made equal in general.

We will favour the first definition of presentability.
\begin{definition}[$y$ presentable object]\label{yonpres}
	Let $y : S \To P$ be a context; $A\in\mathcal K$ is $y$ \emph{presentable} if the following conditions are satisfied:
	\begin{itemize}
		\item \label{p:uno} $A$ is a left split subobject (see \autoref{useful-for-envel}) of $ P G$ for some $G$, via an adjunction $L :  P G \leftrightarrows A : i$;
		\item \label{p:due} $A$ is $y$ accessible (i.e. $A\cong S \overline G$ for some $\overline G$);
		\item \label{p:tre} The functor $i : A\to  P G$ exhibits the left extension of $\alpha_{\overline G}\circ i$ along $\alpha_{\overline G}$ ($\alpha_{\overline G} : \overline G \to S \overline G$ is the unit of the \kz),
		\[
			\vcenter{\xymatrix{
			&\overline G \ar@{}[d]|(.6){\eta\Searrow} \ar[dr]^{\alpha_{\bar G}} \ar[dl]_{\alpha_{\bar G}\circ i}& \\
			 P G && \ar[ll]^i A
			}}
		\]
	\end{itemize}
\end{definition}

\subsection{Faint presentability: non più equivalente alla presentabilità forte}
\label{sec:orgd3d6c82}

The following definition, as well as the notion of \(S\) cell, will reappear when we study \emph{Gabriel-Ulmer structures}, i.e. those Yoneda contexts where the skewness of our two definitions of presentability disappears (this is the content of \autoref{disappears}; in addition, Gabriel-Ulmer structures will turn out to be the right contexts in which we can instantiate a weak form of Gabriel-Ulmer duality in \(\mathcal K\)).
\begin{definition}[Faint presentability]
	Let $y : S\To  P$ be a Yoneda context on $\mathcal K$. An object $A\in\mathcal K$ is called \emph{faintly presentable} if it is a left split subobject of $ P G$ for some $G\in\mathcal K$, and in addition the inclusion $A\to  P G$ is a $S$ cell and $A$ is $S$ cocomplete.
\end{definition}
\begin{prop}
Let \(y\) be a context. Then if \(S G\) is \(y\) presentable, it is also \(y\) faintly presentable.


\subsection{\ldots{}Ma sono equivalenti in un GU-envelope!\ldots{}}
\label{sec:orga7b0bc4}

We start with the simple observation that the closure under finite colimits of \(A\in\Cat\) is the subcategory of \([A^\opp,\Set]\) generated by finite colimits of representables; we call this category \(\widehat A\). It is clear that its opposite \((\widehat{A})^\opp\) has finite \emph{limits}; moreover, we have the following chain of isomorphisms, where \(\Cat_*(\widehat{A}^\opp,\Set)\) is the category of functors \(\widehat{A}^\opp\to \Set\) that commute with finite limits:
\[
	\boldsymbol{Ind}(\widehat{A}) \cong \Cat_*(\widehat{A}^\opp,\Set)\cong [A^\opp,\Set];
\]
in other words, \emph\{there exists an object \(\widehat{A}\) such that\} \(\boldsymbol{Ind}(\widehat A)\cong P A\).

This amounts to a factorization of \(y_A : A \to P A\) as a composition \(A\to \widehat A \to S \widehat A\cong P A\), naturally in \(A\in\cK\). This will turn out to be a fundamental property, in that the definition of a Gabriel\hyp{}Ulmer envelope relative to a context \(y\) amounts to the same ``factorization of \(P\) along \(S\)''. More precisely, we can give the following definition:
\begin{definition}[\gu envelope]\label{guenvelope}
	A \emph{Gabriel\hyp{}Ulmer envelope} (\emph{\gu envelope} for short) relative to a context $y : S\to P$ consists of an additional relative \kz{} denoted $\widehat{ - }$ with unit $\iota_A : A\to \widehat A$ such that $ \alpha_{\hat A} : \widehat A \to S(\widehat A)$ exhibits the left extension of $y_G$ along $\iota_A$. In particular this means that the diagram
	\[
		\vcenter{\xymatrix{
		&A\ar[dr]_{ \alpha_{\hat A}\iota_A}\ar[dl]_{\iota_A}&\\
		\widehat A\ar[rr]_{ \alpha_{\hat A}} && S \widehat A \cong P A
		\ar@/^1.5pc/(15,0);(36,-10)^{y_A}
		}}
	\]
	is filled by an invertible 2-cell $y_A \cong  \alpha_{\hat A}\iota_A$.
\end{definition}
Notice that the 2-category \(\Cat\) has a \gu envelope, relative to the standard context \(\boldsymbol{Ind}_\omega\to [ - ^\opp,\Set]\), defined sending \(A\) into its finite colimit completion \(\widehat A\).
\begin{rmk}
	If $L\dashv R$ is an adjunction in $\Cat$, a well-known sufficient condition so that $L$ preserves $\alpha$-presentable objects is that $R$ commutes with $\alpha$-filtered colimits.
	This simple observation, together with the definition of $S$-cell, motivates the following definition.
\end{rmk}


\subsection{\ldots{}che è esattamente il setting dove vale GU}
\label{sec:org3fcd3b7}

Gabriel\hyp{}Ulmer duality builds an a bi\hyp{}equivalence
\[
	y\textsf{-Mod} : \text{Lex}^\opp \leftrightarrows \text{LFP} : y\textsf{-Th}
\]
between the 2-category of small finitely complete categories, finite limit preserving functors, and natural transformations, and the 2-category \(\one{Lfp}\) of locally finitely presentable categories, finitary right adjoint functors and natural transformations.

The idea is that an object \(C\in\text{Lex}\) is a ``theory'', whose category of models \(\text{Lex}(C,\Set)\) is locally finitely presentable. Gabriel\hyp{}Ulmer duality says that all locally finitely presentable categories arise in this way, as it is possible to extract the theory of which a given \(A\in\text{LFP}\) is the category of models of.
\begin{definition}[The 2-category $\text{Lex}(y)$]
	Let $y : S\To P$ be a Yoneda context on $\cK$, and $\widehat{ - }$ a \gu envelope on $\cK$, that will remain implicit in the discussion. We define the 2-category $\text{Lex}(y)$ having 0-cells the $\widehat{ - }$-cocomplete objects, 1-cells the $\widehat{ - }$-cells, and all 2-cells between them.
\end{definition}
\begin{definition}[The 2-category $\text{LFP}({y})$]
	The objects of the 2-category $\text{LFP}({y})$ are $y$-presentable objects of $\cK$: by our \autoref{the-main}, this class coincides with $y$-accessible and cocomplete 0-cells; 1-cells are right adjoints that are $S$-cells according to \autoref{essecells}, with all 2-cells of $\cK$ between them.
\end{definition}
\begin{remark}
	If $A$ is $y$-presentable, then $A\cong S G$, and $A$ is a reflection on $P (\bar G)$, so that there is a fully faithful right adjoint $L\dashv i : S G\to P(\bar G)$; it is easy to see that we can always reduce to the case where $\bar G = G$: since $P\bar G$ is $P$-cocomplete, the composition $i \circ \alpha_G$ admits a Yoneda extension $I : P G\to P \bar G$, and the composition $L \circ I$ determines a reflection of $P G$ onto $S G \cong A$.
\end{remark}
\begin{theorem}[Gabriel\hyp{}Ulmer duality]\label{guduality}
	Let $y : S\To P$ be a context on $\cK$, with $S$ climbable and assume that there exist an absorbing \gu envelope relative to $y$. Then there is a bi-adjunction
	\[
		y\textsf{-Mod} : \text{Lex}(y)^\opp \rightleftarrows \text{LFP}(y) : y\textsf{-Th}
	\]
	which is in fact a bi-equivalence of 2-categories.
\end{theorem}

\subsection{Esempi, tantissimi esempi}
\label{sec:org1ff6226}

\subsubsection{}
\label{sec:orgccf2ea6}
\subsubsection{}
\label{sec:orga75a3b8}
\subsubsection{}
\label{sec:org084152f}
\subsubsection{}
\label{sec:org6c27358}
\subsubsection{}
\label{sec:orge0541f7}
\subsubsection{}
\label{sec:orgf8d4707}
\subsubsection{}
\label{sec:orgef44d20}
\subsubsection{}
\label{sec:org51d67f8}
\subsubsection{}
\label{sec:orgd9bf7e2}

\subsection{Long term goal: derivatori e infty-categorie}
\label{sec:org794d1ad}
\end{document}