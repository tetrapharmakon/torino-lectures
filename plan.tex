% Created 2018-10-26 ven 17:16
% Intended LaTeX compiler: pdflatex
\documentclass[11pt]{article}
\usepackage[utf8]{inputenc}
\usepackage[T1]{fontenc}
\usepackage{graphicx}
\usepackage{grffile}
\usepackage{longtable}
\usepackage{wrapfig}
\usepackage{rotating}
\usepackage[normalem]{ulem}
\usepackage{amsmath}
\usepackage{textcomp}
\usepackage{amssymb}
\usepackage{capt-of}
\usepackage{hyperref}
\author{Fouche}
\date{\today}
\title{Lectures}
\hypersetup{
 pdfauthor={Fouche},
 pdftitle={Lectures},
 pdfkeywords={},
 pdfsubject={},
 pdfcreator={Emacs 25.2.2 (Org mode 9.1.3)}, 
 pdflang={English}}
\begin{document}

\maketitle
\tableofcontents


\section{Lecture I: Introduzione}
\label{sec:orgb53d0f5}
\subsection{2-categorie: def di partenza}
\label{sec:org9aa8a8e}
\subsection{Estensioni e lift di Kan}
\label{sec:org39136f6}
\subsection{Caratterizzazione degli aggiunti mediante estensioni e lift}
\label{sec:org9220f4a}
\subsection{Il sacro pasting lemma}
\label{sec:orgbd10d2c}
\subsection{Strutture di Yoneda e FCT: motivazioni}
\label{sec:orgd78a6dc}
\subsection{Assiomi di struttura di Yoneda}
\label{sec:org03469d8}
\subsection{La vera natura di P: una KZ monade con opportune proprietà}
\label{sec:org064869f}
\subsection{Teoremi validi nelle strutture di Yoneda}
\label{sec:org9aa5bb5}
\subsection{La nozione di P-cocompletezza}
\label{sec:org388883e}
\subsection{Categorie accessibili e presentabili, classicamente?}
\label{sec:org91b1759}
\section{Lecture II: Accessibility and Presentability in 2-categories}
\label{sec:orge67549c}
\subsection{Cosa vogliamo fare}
\label{sec:orgf93fd49}
Le categorie accessibili e presentabili sono particolari
oggetti della 2-categoria \(\bf CAT\); fino a che punto è possibile
sketchare una definizione per un oggetto
accessibile/presentabile di una 2-categoria \(\mathcal K\)? E' ancora
possibile recuperare i teoremi classici di rappresentazione,
che dicono come gli oggetti accessibili nascano da
riflessioni di oggetti dei prefasci?

E' ancora possibile enunciare e dimostrare la dualità di
Gabriel-Ulmer, ossia la biequivalenza tra la sub-2-categoria
degli oggetti presentabili di K e la 2-categoria degli
"oggetti con limiti finiti"?

\subsection{L'idea per farlo}
\label{sec:org16d78da}
\subsection{Definizione: Yoneda context}
\label{sec:org5b6050a}
\subsection{Definizione: oggetto accessibile wrt un contesto}
\label{sec:org8056442}
\subsection{Definizione: oggetto presentabile wrt un contesto}
\label{sec:orge560ac8}
\subsection{Faint presentability: non più equivalente alla presentabilità forte}
\label{sec:orgcfe39da}
\subsection{\ldots{}Ma sono equivalenti in un GU-envelope!\ldots{}}
\label{sec:orgeab109d}
\subsection{\ldots{}che è esattamente il setting dove vale GU}
\label{sec:org67232a0}
\subsection{Esempi, tantissimi esempi}
\label{sec:org919ea69}
\subsection{Long term goal: derivatori e infty-categorie}
\label{sec:orgda58b7e}
\end{document}