% Created 2018-11-02 ven 23:50
% Intended LaTeX compiler: pdflatex
\documentclass[11pt]{article}
\usepackage[utf8]{inputenc}
\usepackage[T1]{fontenc}
\usepackage{graphicx}
\usepackage{grffile}
\usepackage{longtable}
\usepackage{wrapfig}
\usepackage{rotating}
\usepackage[normalem]{ulem}
\usepackage{amsmath}
\usepackage{textcomp}
\usepackage{amssymb}
\usepackage{capt-of}
\usepackage{hyperref}
\def\C{\mathbf{C}}
\def\Cat{\mathsf{Cat}}
\def\leeft{\text{lift}}
\def\xto#1{\xrightarrow{#1}}\usepackage[all,2cell]{xy}
\newcommand{\deduction}[4]{
\begin{array}{c}
#1 \to #2 \\ \hline
#3 \to #4
\end{array}
}
\newcommand{\Nearrow}{\rotatebox[origin=c]{45}{$\Rightarrow$}}  % ↗
\newcommand{\Nwarrow}{\rotatebox[origin=c]{135}{$\Rightarrow$}} % ↖
\newcommand{\Searrow}{\rotatebox[origin=c]{-45}{$\Rightarrow$}} % ↘
\newcommand{\Swarrow}{\rotatebox[origin=c]{225}{$\Rightarrow$}} % ↙
\newcommand{\Sarrow}{\rotatebox[origin=c] {-90}{$\Rightarrow$}}
\newcommand{\Narrow}{\rotatebox[origin=c] {90}{$\Rightarrow$}}
\usepackage{turnstile}
\newcommand{\adjunct}[2]{\nsststile{#2}{#1}}
\def\opp{\op}
\def\co{\co}
\def\coop{\text{coop}}
\def\rift{\text{rift}}
\def\leeft{\text{lift}} % `lift is already something!
\def\lan{\text{lan}}
\def\ran{\text{ran}}
\def\Rift{\text{Rift}}
\def\Lift{\text{Lift}}
\def\Ran{\text{Ran}}
\def\Lan{\text{Lan}}
\def\RIFT{\textsc{rift}}
\def\LIFT{\textsc{lift}}
\def\RAN{\textsc{ran}}
\def\LAN{\textsc{lan}}
\usepackage{amsthm}
\theoremstyle{reference}
\newtheorem{theorem}{Theorem}[section]
\newtheorem{conjec}[theorem]{Conjecture}
\newtheorem{corollary}[theorem]{Corollary}
\newtheorem{counterex}[theorem]{Counterexample}
\newtheorem{definition}[theorem]{Definition}
\newtheorem{example}[theorem]{Example}
\newtheorem{exercise}[theorem]{Exercise}
\newtheorem{lemma}[theorem]{Lemma}
\newtheorem{notat}[theorem]{Notation}
\newtheorem{proposition}[theorem]{Proposition}
\newtheorem{question}[theorem]{Question}
\newtheorem{remark}[theorem]{Remark}
\newtheorem{scholium}[theorem]{Scholium}
\newtheorem{setting}[theorem]{Setting}
\newtheorem{conjecture}[theorem]{Conjecture}
\author{Fosco Loregian}
\date{\today}
\title{Strutture di Yoneda e accessibilità}
\hypersetup{
 pdfauthor={Fosco Loregian},
 pdftitle={Strutture di Yoneda e accessibilità},
 pdfkeywords={},
 pdfsubject={},
 pdfcreator={Emacs 25.2.2 (Org mode 9.1.3)}, 
 pdflang={English}}
\begin{document}

\maketitle
\tableofcontents




\section{Lecture I: Introduzione}
\label{sec:orgcba1c6d}
\subsection{2-categorie: def di partenza}
\label{sec:orgf326424}
Una \emph{2-categoria} è ``come una categoria, ma gli hom-set
sono categorie'': si tratta di un certo tipo di struttura
2-dimensionale che soddisfa le seguenti ipotesi

\begin{enumerate}
\item E' data una classe di oggetti \(\mathcal K_0\),
\item E' dato per ogni coppia di oggetti \(X,Y\in \cal K_0\) un
insieme di morfismi (o \emph{1-celle}) \({\cal K}(X,Y)\)
\item E' data una regola di composizione
\end{enumerate}
$$ c_{XYZ} : {\cal K}(X,Y)\times {\cal K}(Y,Z) \to {\cal
K}(X,Z) $$ che sia \emph{bifuntoriale}, ossia tale che valga la
``regola di interscambio di Godement'': 

$$ ... $$ 

Una 2-categoria è una categoria arricchita su \(\Cat\),
guardata come base monoidale (chiusa): ogni \(\Cat({\cal
C},{\cal D})\) è a sua volta una categoria, i cui oggetti
sono i funtori \(F,G : \C\to {\cal D}\) e i cui morfismi
sono le trasformazioni naturali. Ciascuna composizione è
``bilineare'', ed esiste una nozione di funtore arricchito
(\emph{2-funtore stretto}) e di trasformazione naturale
arricchita (\emph{trasf. naturale stretta}).

La teoria delle 2-categorie coincide allora con la teoria
delle \(\Cat\) -categorie? In parte sì: un gran numero di
risultati è conseguenza della teoria generale sviluppata nel
primo capitolo. Del resto, una parte ancora maggiore di
risultati non si può ingabbiare nel linguaggio delle
categorie arricchite: e questo perché segretamente la
collezione delle 2-categorie (di cui, fatti salvi alcuni
problemi di teoria degli insiemi, \(\Cat\) è un oggetto) è una
\textbf{3-categoria}; le sue proprietà sono quindi più naturalmente
descritte da strutture e leggi di coerenza più blande di
quelle che sostengono la teoria delle \(\Cat\) -categorie.

Un esempio su tutti: nei termini di una \(\Cat\) -categoria è
complicato (o impossibile?) descrivere come faccia un
diagramma a commutare \textbf{a meno} di una trasformazione
naturale -invertibile o meno.

Alcune osservazioni:

\begin{itemize}
\item Ogni 2-categoria dà luogo ad altre 2-categorie \(\mathcal
  K^\co, \mathcal K^\op\) dove rispettivamente
sono state invertite le frecce in dimensione 2 ed 1. Più
formalmente, \(\mathcal K^\co\) è una 2-categoria
ottenuta da \(\mathcal K\) che ha gli stessi oggetti e dove
\(\mathcal K^\co(X,Y) = \mathcal K(X,Y)^\op\), e
\(\mathcal K^\op\) è una 2-categoria con gli stessi
oggetti di \(\mathcal K\), dove \(\mathcal K^\op(X,Y) =
  \mathcal K(Y,X)\). Chiaramente, \((\mathcal
  K^\op)^\co = (\mathcal K^\co)^\op
  = \mathcal K^\text{coop}\).
\end{itemize}

\subsection{Estensioni e lift di Kan}
\label{sec:org25a80de}

\textbf{Definizione.} Let \(B \xto{f} A \xot{g}C\) a cospan of
1-cells in \({\mathcal K}\). A \emph{left lifting} of \(f\) along \(g\)
consists of a pair \(\langle\leeft_gf,\eta\rangle\) (often
denoted simply as \(\leeft_gf\)) initial among the commutative
triangles like the one below: 
\[
\vcenter{\xymatrix@C=1.4cm{& C\ar[d]^g \\ B\ar[r]_f
\ar@{.>}[ur]^{\leeft_gf} & \ar@{}[ul]|(.3){\Nearrow\eta} A}}
\qquad \deduction{\leeft_gf}{h}{f}{gh} 
\] In other words,
composition with \(\eta \colon f \To g \circ \leeft_gf\)
determines a bijection \(\bar\gamma \mapsto (g *
\bar\gamma)\circ \eta\) between 2-cells \(\leeft_gf
\xto{\bar\gamma} h\) and 2-cells \(f \to gh\).

One can define \emph{right liftings} similarly, reversing
only the direction of the 2-cell in the diagram above, and
consequently the universal property, and \emph{left} and
\emph{right extensions} reversing, respectively, only the
directions of 1-cells or the direction of both 1- and
2-cells in the diagram above. It is then clear that left
extensions in \({\mathcal K}\) are left liftings in \({\mathcal
K}^\opp\), right liftings in \({\mathcal K}\) are left liftings
in \({\mathcal K}^\co\), and right extensions are left
liftings in \({\mathcal K}^\coop\).

The situation is conveniently depicted in the following array of universal
objects:

\begin{center}
\begin{array}{|c|c|}\hline \xymatrix{A \ar@{}[dr]|(.3){\Swarrow\eta}\ar[d]_g
\ar[r]^f& B \\ C \ar@{.>}[ur]_{\Lan_gf} & {\tiny \deduction{\Lan_gf}{h}{f}{hg}}}
& \xymatrix{{\tiny \deduction{\Lift_gf}{h}{f}{gh}} & C\ar[d]^g \\ B\ar[r]_f
\ar@{.>}[ur]^{\Lift_gf} & \ar@{}[ul]|(.3){\Nearrow\eta} A} \\ \hline
%%%
\xymatrix{A \ar@{}[dr]|(.3){\Nearrow\varepsilon}\ar[d]_g \ar[r]^f& B \\ C
\ar@{.>}[ur]_{\Ran_gf} & {\tiny \deduction{hg}{f}{h}{\Ran_gf}}} &
\xymatrix{{\tiny \deduction{h}{\Rift_gf}{gH}{f}} & C\ar[d]^g \\ B\ar[r]_f
\ar@{.>}[ur]^{\Rift_gf} & \ar@{}[ul]|(.3){\Swarrow\varepsilon} A} \\ \hline
\end{array}
\end{center}

\subsection{Caratterizzazione degli aggiunti mediante estensioni e lift}
\label{sec:org26b90bd}

Qui ci proponiamo di dimostrare una caratterizzazione degli
aggiunti in termini di lift ed estensioni. Si tratta di un
esercizio elementare nelle proprietà universali in una
2-categoria, che svolgiamo nei dettagli per fare entrare il
lettore nella semantica delle 2-categorie.

\begin{prop}
ABCde f
\end{prop}

\subsection{Il sacro pasting lemma}
\label{sec:org46eb4aa}
\subsection{Strutture di Yoneda e FCT: il lemma di Yoneda come pilastro del pensiero occidentale}
\label{sec:org4f6eee9}
\subsubsection{Yoneda nel senso classico; Yoneda "per i baby geometri"}
\label{sec:org48a76f8}
\subsubsection{Di cosa parliamo quando parliamo di teoria delle categorie?}
\label{sec:org12570a5}
\subsection{Assiomi di struttura di Yoneda}
\label{sec:orga3a215c}
\subsection{La vera natura di P: una KZ monade con opportune proprietà}
\label{sec:org8a07368}
\subsection{Teoremi validi nelle strutture di Yoneda}
\label{sec:orgb9f0a24}
\subsection{La nozione di P-cocompletezza}
\label{sec:orgc9e386a}
\subsection{Categorie accessibili e presentabili, classicamente?}
\label{sec:orgb7f855d}
\section{Lecture II: Accessibility and Presentability in 2-categories}
\label{sec:orge7b98a1}
\subsection{Cosa vogliamo fare}
\label{sec:org4881c98}
Le categorie accessibili e presentabili sono particolari
oggetti della 2-categoria \(\bf CAT\); fino a che punto è possibile
sketchare una definizione per un oggetto
accessibile/presentabile di una 2-categoria \(\mathcal K\)? E' ancora
possibile recuperare i teoremi classici di rappresentazione,
che dicono come gli oggetti accessibili nascano da
riflessioni di oggetti dei prefasci?

E' ancora possibile enunciare e dimostrare la dualità di
Gabriel-Ulmer, ossia la biequivalenza tra la sub-2-categoria
degli oggetti presentabili di K e la 2-categoria degli
"oggetti con limiti finiti"?

\subsection{L'idea per farlo}
\label{sec:org9dbae78}

Utilizzare il linguaggio delle strutture di Yoneda; il teorema classico di "rappresentazione" che dice che una categoria è presentabile se e solo se 

\subsection{Definizione: Yoneda context}
\label{sec:orgb4a6cc9}
\subsection{Definizione: oggetto accessibile wrt un contesto}
\label{sec:org4f56f29}
\subsection{Definizione: oggetto presentabile wrt un contesto}
\label{sec:org15c0a19}
\subsection{Faint presentability: non più equivalente alla presentabilità forte}
\label{sec:orgaeee656}
\subsection{\ldots{}Ma sono equivalenti in un GU-envelope!\ldots{}}
\label{sec:orgf5c84f1}
\subsection{\ldots{}che è esattamente il setting dove vale GU}
\label{sec:org3ef6adb}
\subsection{Esempi, tantissimi esempi}
\label{sec:org6b79e8f}
\subsection{Long term goal: derivatori e infty-categorie}
\label{sec:org701e285}
\end{document}